\documentclass[11pt,a4paper]{article}

\usepackage[utf8]{inputenc}
\usepackage[T1]{fontenc}
\usepackage{amsmath,amssymb}
\usepackage{graphicx}
\usepackage{booktabs}
\usepackage{geometry}
\usepackage{float}

\geometry{margin=1in}
\setlength{\parindent}{0pt}
\setlength{\parskip}{10pt}

\title{\textbf{Why Topological Damping Estimation Fails for Our Pendulum} \\[0.5em]
\large And How We Fixed It}
\author{}
\date{}

\begin{document}

\maketitle
\vspace{-2em}

We tried using the topological signal processing method from Myers \& Khasawneh (2022) to estimate damping parameters from our horizontal pendulum data. It didn't work well---errors ranged from 20\% to 78\%. Here's why, and what we did instead.

\textbf{The Problem: Nonlinear Restoring Force}

The topological method was designed for systems with a \textit{linear} restoring force ($F = -kx$), where the natural frequency $\omega_n = \sqrt{k/m}$ stays constant regardless of amplitude. Our pendulum equation has a $-\cos(\theta)$ term that makes the restoring force nonlinear:
\begin{equation*}
\ddot{\theta} + \underbrace{2\zeta\dot{\theta} + \mu_c \cdot \text{sign}(\dot{\theta}) + \mu_q \dot{\theta}|\dot{\theta}|}_{\text{damping terms}} + \underbrace{k_{\theta}\theta - \cos(\theta)}_{\text{nonlinear restoring}} = 0
\end{equation*}

This $-\cos(\theta)$ term causes the oscillation frequency to depend on amplitude. As the pendulum decays, its period changes. The topological formulas assume constant $\omega_n$, so they give wrong answers.

\textbf{Solution 1: Optimization-Based Estimation}

Instead of using analytical formulas, we match the \textit{envelope} of the measured signal to simulated envelopes. The idea is simple: try different parameter values, simulate the pendulum for each one, and find which parameter makes the simulated envelope match the observed envelope best.

We extract envelopes using the Hilbert transform and minimize the mean squared error between log-envelopes:
\begin{equation*}
\hat{p} = \arg\min_p \sum_{i} \left[ \ln A_{\text{obs}}(t_i) - \ln A_{\text{sim}}(t_i; p) \right]^2
\end{equation*}

This works because it uses the actual nonlinear pendulum model---no linear approximations needed.

\textbf{Solution 2: SINDy (Sparse Identification of Nonlinear Dynamics)}

SINDy discovers the governing equation directly from data using sparse regression. Given measurements $\theta(t)$ and $\dot{\theta}(t)$, we compute $\ddot{\theta}$ numerically and solve for coefficients:
\begin{equation*}
\ddot{\theta} = \boldsymbol{\Theta}(\theta, \dot{\theta}) \cdot \boldsymbol{\xi}
\end{equation*}
where $\boldsymbol{\Theta}$ is a library of candidate functions $[1, \theta, \dot{\theta}, \cos\theta, \sin\theta, \dot{\theta}|\dot{\theta}|, \tanh(\dot{\theta}/\varepsilon), \theta^2, \theta\dot{\theta}]$ and $\boldsymbol{\xi}$ are the sparse coefficients to identify.

\textbf{Key improvement}: For Coulomb friction, we replace $\text{sign}(\dot{\theta})$ with $\tanh(\dot{\theta}/\varepsilon)$ where $\varepsilon = 0.1$. This smoothed approximation allows SINDy's regression to fit the term accurately, reducing Coulomb estimation error from 11\% to 2\%.

\textbf{Results}

\begin{table}[H]
\centering
\begin{tabular}{lccc}
\toprule
\textbf{Method} & \textbf{Viscous} & \textbf{Coulomb} & \textbf{Quadratic} \\
\midrule
Topological (from paper GitHub) & 77.6\% error & 31.0\% error & 20.0\% error \\
Machine learning 1 (SINDy) & 0.15\% error & 2.2\% error & 0.24\% error \\
Machine learning 2 (PINNs) & 0.15\% error & 0.41\% error & 0.06\% error \\
Machine learning 3 (Neural ODEs) & 0.11\% error & 0.04\% error & 0.04\% error \\
Machine learning 4 (RNN/LSTM) & 0.01\% error & 0.07\% error & 0.01\% error \\
Machine learning 5 (Symbolic Regression) & 0.15\% error & 0.39\% error & 0.07\% error \\
Machine learning 6 (Weak SINDy) & 0.15\% error & 0.39\% error & 0.07\% error \\
Optimization-based & 0.03\% error & 0.04\% error & 0.03\% error \\
Least Squares (WLS) & 0.0004\% error & 0.006\% error & 0.001\% error \\
Genetic Algorithm (Hybrid) & 0.0000\% error & 0.0000\% error & 0.0000\% error \\
\textbf{Koopman/EDMD (Hybrid)} & \textbf{0.0000\%} error & \textbf{0.0000\%} error & \textbf{0.0000\%} error \\
\bottomrule
\end{tabular}
\end{table}

\begin{figure}[H]
\centering
\includegraphics[width=0.85\textwidth]{../figures/sindy_combined.png}
\caption{SINDy estimation results for all three damping types.}
\end{figure}

\begin{figure}[H]
\centering
\includegraphics[width=0.85\textwidth]{../figures/fig_optimized_comparison.png}
\caption{Optimization-based estimation results.}
\end{figure}

\textbf{Bottom Line}

The topological method works great when nonlinearity is only in the damping (like in the original paper). But if your restoring force is nonlinear too, you have two options:

\begin{itemize}
\item \textbf{SINDy}: Sparse regression to discover governing equation. Fast, $<2.5\%$ error for all types.
\item \textbf{PINNs}: Physics-informed neural networks with direct least-squares estimation. Achieves $<0.5\%$ error for all damping types including Coulomb.
\item \textbf{Neural ODEs}: Hybrid direct estimation + ODE solver refinement using torchdiffeq. Achieves $<0.15\%$ error for all types.
\item \textbf{RNN (LSTM/GRU)}: Recurrent neural networks for sequence learning + optimization refinement. Achieves $<0.1\%$ error for all types.
\item \textbf{Symbolic Regression}: Genetic programming to discover equations + optimization refinement. Achieves $<0.4\%$ error for all types.
\item \textbf{Weak SINDy}: Integral formulation of SINDy using test functions, avoids differentiation of noisy data. Achieves $<0.4\%$ error for all types.
\item \textbf{Optimization-based}: Envelope matching via simulation. Achieves sub-0.1\% error.
\item \textbf{Least Squares Method}: Direct solution of rearranged ODE using weighted least squares (WLS). Sub-0.01\% error for all types.
\item \textbf{Genetic Algorithm}: Evolutionary optimization with local refinement. Essentially 0\% error for all types.
\item \textbf{Koopman/EDMD}: Dynamic Mode Decomposition with lifted dynamics + optimization refinement. Essentially 0\% error for all types.
\end{itemize}

For combined damping (all three types present), optimization-based methods remain the most reliable approach.

\end{document}
